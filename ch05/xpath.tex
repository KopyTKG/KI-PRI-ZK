\chapter{xPath}
\label{cha:xPath}

XPath, zkratka pro XML Path Language, je jazyk pro výběr uzlů z XML dokumentů. XPath umožňuje navigovat přes elementy a atributy v XML dokumentech, což je užitečné pro různé účely, jako je extrakce informací, manipulace s daty nebo při transformacích XML pomocí XSLT. XPath definuje cestu k určitému uzlu nebo skupině uzlů a používá se pro jednoduché až složité dotazy na strukturu XML dokumentu.

\section{Správa}
\label{sec:xPath_spr}
XPath je spravován W3C (World Wide Web Consortium), což je mezinárodní komunita, která vyvíjí otevřené standardy pro zajištění dlouhodobého růstu Webu. W3C udržuje a aktualizuje specifikace XPath, aby byly kompatibilní s ostatními technologiemi a vyhovovaly novým potřebám uživatelů a vývojářů.

\section{Struktura XPath}
XPath využívá cesty podobné adresám ve filesystemech k lokalizaci informací v XML dokumentech. Několik základních syntaktických prvků zahrnuje:

\begin{itemize}
  \item \textbf{/} - označuje kořenový uzel dokumentu a zahajuje absolutní cestu.
  \item \textbf{//} - vyhledává uzly v dokumentu od aktuálního uzlu, které odpovídají vzoru bez ohledu na jejich hloubku.
  \item \textbf{.} - reprezentuje aktuální uzel.
  \item \textbf{..} - reprezentuje rodičovský uzel aktuálního uzlu.
  \item \textbf{@} - používá se pro výběr atributů.
\end{itemize}
