\chapter{PHP} % (fold)
\label{cha:PHP}

PHP je serverový skriptovací jazyk, který se široce používá pro vývoj webových aplikací. PHP je zkratka pro "PHP: Hypertext Preprocessor" a byl navržen tak, aby snadno integroval skriptování na straně serveru s HTML. Umožňuje vývojářům vytvářet dynamické webové stránky, které mohou interagovat s databázemi a provádět komplexní funkce.

\section{Základy}
PHP kód se píše ve skriptech umístěných mezi tagy <?php a ?>. Kód se spustí na serveru, který generuje HTML, jež je odesíláno do webového prohlížeče. Toto umožňuje vytvářet personalizovaný obsah pro uživatele.

\section{Struktury}
PHP podporuje běžné programovací struktury, včetně podmínek (if, else), smyček (for, while, foreach) a funkcí. Tyto struktury umožňují vývojářům psát dobře organizovaný a opakovaně použitelný kód.

\section{Formuláře}
PHP je obzvláště užitečné pro zpracování dat z HTML formulářů. Data odeslaná formulářem mohou být přijata PHP skriptem a použita pro různé účely, jako je ukládání informací do databáze nebo validace uživatelských vstupů.

\section{Pole a proměnné}
Proměnné v PHP jsou uvozeny znakem \$ a mohou obsahovat širokou škálu datových typů, včetně řetězců, celých čísel a polí. Pole mohou být indexovaná nebo asociativní, kde indexovaná pole používají číselné indexy a asociativní používají názvy klíčů.

\section{Kdo to spravuje}
PHP je open-source a je spravováno PHP Group, která zodpovídá za vývoj jazyka. PHP Group zveřejňuje aktualizace a udržuje dokumentaci, která je dostupná na oficiálních webových stránkách PHP.
\begin{figure}[H]
\centering
\begin{lstlisting}[
           language=php,
           showspaces=false,
           basicstyle=\ttfamily,
           commentstyle=\color{gray},
           keywordstyle=\color{cyan}
]
<?php
// Získání dat z formulare
$jmeno = $_POST['jmeno'];
$email = $_POST['email'];

// Zobrazení dat
echo "Jméno:$jmeno<br>";
echo "Email:$email<br>";
?>
\end{lstlisting}
\caption{Ukázka php kódu}
\label{fig:php_file}
\end{figure}


\section{PHP Objekty}
PHP objekty jsou instance tříd, které jsou základními stavebními kameny objektově orientovaného programování (OOP) v PHP. OOP umožňuje programátorům organizovat kód do modulárních, znovupoužitelných jednotek zvaných třídy, které definují vlastnosti (atributy) a chování (metody) objektů.

\subsection{Definice a použití}
Objekty jsou vytvořeny pomocí klíčového slova \texttt{new} a jména třídy. Například:

\begin{figure}[H]
\centering
\begin{lstlisting}[
           language=php,
           showspaces=false,
           basicstyle=\ttfamily,
           commentstyle=\color{gray},
           keywordstyle=\color{cyan}
]
class Osoba {
    public $jmeno;
    public $vek;

    public function __construct($jmeno, $vek) {
        $this->jmeno = $jmeno;
        $this->vek = $vek;
    }
    
    public function pozdrav() {
        return "Ahoj, jmenuji se " . $this->jmeno;
    }
}

$osoba = new Osoba("Petr", 25);
echo $osoba->pozdrav();
\end{lstlisting}
\caption{PHP objekt}
\label{fig:php_object}
\end{figure}

\section{PHP Vyjímky}
Vyjímky jsou způsob, jakým PHP zpracovává chyby v objektově orientovaném programování. Vyjímky umožňují programu přerušit normální tok vykonávání a vykonat kód určený k ošetření chyby.

\subsection{Druhy a dělení}
Ve výchozím nastavení PHP poskytuje několik základních tříd vyjímek, jako je \texttt{Exception} a \texttt{ErrorException}. Vývojáři mohou vytvářet vlastní vyjímky odvozením od těchto základních tříd. Vyjímky mohou být rozděleny podle úrovně závažnosti nebo podle komponenty, ve které se chyba objeví.

\subsection{Použití vyjímek}
Pro zachycení vyjímek se používá blok \texttt{try-catch}. Během vykonávání bloku \texttt{try} může být vyvolána vyjímka, která je následně zachycena v bloku \texttt{catch}, kde je možné ji zpracovat.

\begin{figure}[H]
\centering
\begin{lstlisting}[
           language=php,
           showspaces=false,
           basicstyle=\ttfamily,
           commentstyle=\color{gray},
           keywordstyle=\color{cyan}
]
try {
    $cislo = 0;
    if($cislo == 0) {
        throw new Exception("Deleni nulou.");
    }
    $vysledek = 100 / $cislo;
} catch (Exception $e) {
    echo "Chyba:" . $e->getMessage();
}
\end{lstlisting}
\caption{PHP vyjímky}
\label{fig:php_exeptions}
\end{figure}

\section{PHP a XMLdom} 
\label{cha:PHP_XMLdom}

PHP umožňuje dynamickou manipulaci s XML pomocí XML DOM. XML DOM v PHP poskytuje rozhraní pro práci s XML dokumenty jako s objektovými stromy.

\begin{figure}[H]
\centering
\begin{lstlisting}[
           language=php,
           showspaces=false,
           basicstyle=\ttfamily,
           commentstyle=\color{gray},
           keywordstyle=\color{cyan}
]
<?php
$doc = new DOMDocument();
$doc->load('example.xml');
$root = $doc->documentElement;
$elements = $root->getElementsByTagName('element');
foreach ($elements as $el) {
    echo $el->nodeValue;
}
?>
\end{lstlisting}
\caption{Ukázka xml a php}
\label{fig:xml_schema}
\end{figure}