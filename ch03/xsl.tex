\chapter{XSL}
\label{cha:XSL}
XSL zahrnuje XSLT (XSL Transformations), XPath (XML Path Language) a XSL-FO (XSL Formatting Objects), které umožňují transformovat, vyhledávat a formátovat XML data pro různé výstupní formáty, jako je HTML, text, nebo PDF.

XSLT je pravděpodobně nejčastěji používaná část XSL a slouží k transformaci XML dokumentů do jiných formátů (např. HTML) pomocí šablonového přístupu. Tento proces transformace umožňuje zobrazit XML data ve webových prohlížečích nebo je integrovat do jiných aplikací.
\begin{figure}[H]
\centering
\begin{lstlisting}[
           language=xml,
           showspaces=false,
           basicstyle=\ttfamily,
           commentstyle=\color{gray},
           keywordstyle=\color{cyan}
]
<?xml version="1.0" encoding="UTF-8"?>
<xsl:stylesheet 
    version="1.0" 
    xmlns:xsl="http://www.w3.org/1999/XSL/Transform">
  <xsl:template match="/">
    <html>
      <body>
        <h2>Seznam knih</h2>
        <table border="1">
          <tr>
            <th>Název</th>
            <th>Autor</th>
          </tr>
          <xsl:for-each select="knihovna/kniha">
            <tr>
              <td><xsl:value-of select="nazev"/></td>
              <td><xsl:value-of select="autor"/></td>
            </tr>
          </xsl:for-each>
        </table>
      </body>
    </html>
  </xsl:template>
</xsl:stylesheet>
\end{lstlisting}
\caption{Ukázka XSL souboru}
\label{fig:xsl_file}
\end{figure}

\subsection{Šablony a Výběry (Templates and Matching)}
\begin{itemize}
    \item \textbf{`<xsl:template match="...">`} - Definuje pravidla pro zpracování určitých částí XML dokumentu. Atribut 'match' umožňuje specifikovat XPath výrazy, které identifikují elementy nebo atributy pro aplikaci šablony.
\end{itemize}

\subsection{Podmíněné Zpracování (Conditional Processing)}
\begin{itemize}
    \item \textbf{`<xsl:if test="...">`} - Provádí test, a pokud je výsledek pravdivý, aplikuje určitý kód.
    \item \textbf{`<xsl:choose>`, `<xsl:when>`, a `<xsl:otherwise>`} - Slouží pro vytváření komplexnějších podmíněných struktur, kde '<xsl:choose>' funguje jako kontejner pro jednu nebo více '<xsl:when>' větví a volitelný '<xsl:otherwise>' pro výchozí chování.
\end{itemize}

\subsection{Iterace (Looping)}
\begin{itemize}
    \item \textbf{`<xsl:for-each select="...">`} - Iteruje přes množinu uzlů specifikovanou XPath výrazem v 'select' a aplikuje vložené šablony na každý uzel.

\end{itemize}

\subsection{Řazení (Sorting)}
\begin{itemize}
    \item \textbf{`<xsl:sort select="..." order="ascending|descending" data-type="text|number">`} - Používá se uvnitř '<xsl:for-each>' nebo '<xsl:apply-templates>' pro řazení uzlů podle textu nebo číselných hodnot.
\end{itemize}

\subsection{Vkládání Hodnot (Outputting Values)}
\begin{itemize}
    \item \textbf{`<xsl:value-of select="...">`} - Extrahuje text z vybraného uzlu a vkládá ho do výstupního dokumentu.
    \item \textbf{`<xsl:copy-of select="...">`} - Kopíruje celé uzly z vstupního dokumentu do výstupu, včetně všech dětských uzlů.
\end{itemize}

\subsection{Tvorba a Použití Proměnných (Variables and Parameters)}
\begin{itemize}
    \item \textbf{`<xsl:variable name="..." select="...">`} - Definuje proměnnou, která může být použita v rámci šablony.
    \item \textbf{`<xsl:param name="...">`} - Definuje parametr, který může být předán do šablony z vnějšího kontextu nebo jiné šablony.
\end{itemize}

\subsection{Rekurze (Recursion)}
\begin{itemize}
    \item \textbf{`<xsl:apply-templates select="...">`} - Aplikuje šablony na vybrané uzly, což může zahrnovat rekurzivní volání na uzly, které samy obsahují další uzly pro zpracování.
\end{itemize}

