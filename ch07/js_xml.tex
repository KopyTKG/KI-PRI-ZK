\chapter{Javascript a XML DOM} % (fold)
\label{cha:Javascript_a_XML_DOM}

JavaScript je dynamický skriptovací jazyk, který se používá pro vytváření interaktivních webových stránek a aplikací. XML DOM (Document Object Model) je programové rozhraní, které umožňuje skriptům, jako je JavaScript, dynamicky přistupovat k a manipulovat s XML dokumenty. XML DOM reprezentuje XML dokument jako strom objektů, což umožňuje programátorům jednoduše změnit strukturu dokumentu, obsah nebo styl. JavaScript spolu s XML DOM poskytuje výkonné nástroje pro práci s XML daty na klientově straně, což umožňuje například zobrazování, úpravu, validaci nebo přenos XML dat mezi klientem a serverem bez potřeby obnovovat celou stránku.

\section{Správa}
XML DOM je spravován World Wide Web Consortium (W3C), což je mezinárodní komunita, která vyvíjí otevřené webové standardy, aby zabezpečila dlouhodobý růst webu. W3C udržuje specifikace XML, XML DOM a mnoho dalších technologií, které jsou klíčové pro interoperabilitu mezi webovými technologiemi.

\section{Struktura XML DOM}

XML DOM reprezentuje XML dokumenty jako strom objektů, kde každý uzel v stromu odpovídá určité části dokumentu. Následuje stručný popis hlavních typů uzlů:

\begin{itemize}
  \item \textbf{Element Nodes} - Reprezentují elementy XML a mohou obsahovat další uzly, jako jsou text, další elementy, nebo atributy.
  \item \textbf{Text Nodes} - Obsahují textový obsah elementů.
  \item \textbf{Attribute Nodes} - Uchovávají atributy příslušných elementů, jsou přístupné prostřednictvím elementů, které je obsahují.
  \item \textbf{Comment Nodes} - Umožňují vložení komentářů do XML dokumentů.
\end{itemize}

\section{Manipulace s uzly pomocí JavaScriptu}

JavaScript umožňuje manipulaci s uzly v XML DOM, což zahrnuje přidávání, mazání a modifikaci uzlů, stejně jako změnu jejich atributů. Toto je základní přístup k dynamické interakci s XML na webových stránkách.

