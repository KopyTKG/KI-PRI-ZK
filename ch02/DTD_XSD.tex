\chapter{DTD a XSD}
\label{cha:dtd_xsd}

\section{DTD}
\label{sec:dtd}
DTD (Document Type Definition), což je jeden z prvních a základních mechanismů pro definování struktury a pravidel pro XML dokumenty. DTD umožňuje specifikovat, které elementy, atributy a entity mohou v dokumentu existovat, jaké jsou jejich vztahy a jak často se mohou v dokumentu objevit. Toto je klíčové pro zajištění konzistence dat mezi různými systémy a pro validaci struktury XML dokumentů před jejich zpracováním.

DTD je obvykle zapsáno buď přímo v XML dokumentu, nebo jako externí soubor, který je poté odkazován z XML. K definování pravidel v DTD se používají deklarace elementů a atributů. Například, můžete definovat, že každý dokument musí obsahovat kořenový element <knihovna>, který může obsahovat několik <kniha> elementů, přičemž každý <kniha> element musí obsahovat <nazev>, <autor>, a <vydano>.

\subsection{Datové typy obsahu}
\begin{enumerate}
    \item \textbf{\#PCDATA} - Znamená "Parsed Character Data", což umožňuje vložení běžného textu. Tento typ se používá, když chcete, aby element obsahoval jen text.
    \item \textbf{Prázdný} - Pomocí klíčového slova EMPTY můžete specifikovat, že element nesmí obsahovat žádný obsah ani žádné další elementy.
    \item \textbf{Smíšený obsah} - Můžete definovat element, který může obsahovat kombinaci textu a dalších elementů. To se specifikuje pomocí modelu obsahu kombinujícího \#PCDATA s názvy dalších elementů.
\end{enumerate}

\subsection{Datové typy atributů}
\begin{enumerate}
    \item \textbf{CDATA} - Pro běžné textové řetězce.
    \item \textbf{ID} - Unikátní identifikátor v rámci dokumentu.
    \item \textbf{IDREF/IDREFS} - Odkaz na element s typem ID.
    \item \textbf{NMTOKEN/NMTOKENS} - Token, který musí odpovídat XML názvovým konvencím (např. musí začínat písmenem).
    \item \textbf{ENUMERATION} - Umožňuje definovat seznam povolených hodnot.
    \item \textbf{NMOTATION} - Odkaz na notaci definovanou v DTD.
\end{enumerate}


\begin{figure}[H]
\centering
\begin{lstlisting}[
           language=xml,
           showspaces=false,
           basicstyle=\ttfamily,
           commentstyle=\color{gray},
           keywordstyle=\color{cyan}
]
<!DOCTYPE knihovna [
<!ELEMENT knihovna (kniha+)>
<!ELEMENT kniha (nazev, autor, vydano)>
<!ELEMENT nazev (#PCDATA)>
<!ELEMENT autor (#PCDATA)>
<!ELEMENT vydano (#PCDATA)>
]>

\end{lstlisting}
\caption{Ukázka DTD souboru}
\label{fig:dtd_file}
\end{figure}

\section{XSD}
\label{sec:XSD}
XSD je mocný nástroj pro definování struktury a validaci XML dokumentů. Na rozdíl od DTD poskytuje XSD podrobnější specifikace pro datové typy, podporuje jmenné prostory a umožňuje vytváření složitých a omezujících pravidel pro XML dokumenty.
XSD se používá pro definování pravidelné struktury XML dokumentů, což zajišťuje, že data splňují definované formáty a omezení. To je nezbytné pro aplikace, které vyžadují vysokou úroveň integrity dat, jako jsou finanční systémy, zdravotnické informační systémy, a webové služby, kde je důležitá správná a konzistentní interpretace dat mezi různými systémy a platformami.
Hlavní kategorie datových typů
\begin{enumerate}
    \item \textbf{Primitivní datové typy} - To jsou základní typy, ze kterých jsou odvozeny všechny ostatní typy.
    \item \textbf{Odvozené datové typy} - Tyto typy jsou odvozeny z primitivních typů a mohou zahrnovat dodatečná omezení.
    \item \textbf{Speciální datové typy}
\end{enumerate}
\subsection{Primitivní datové typy}
\begin{itemize}
    \item \textbf{`xs:string`} - Pro textové řetězce.
    \item \textbf{`xs:boolean`} - Pro pravdivostní hodnoty (true/false).
    \item \textbf{`xs:decimal`, `xs:integer`, `xs:float`, `xs:double`} - Pro číselné hodnoty, kde `xs:decimal` a `xs:integer` jsou pro celá a desetinná čísla bez ztráty přesnosti, zatímco `xs:float` a `xs:double` jsou pro plovoucí desetinné čárky s možnou ztrátou přesnosti.
    \item \textbf{`xs:date`, `xs:time`, `xs:dateTime`, `xs:duration`} - Pro datum, čas a jejich kombinace.
\end{itemize}

\subsection{Odvozené datové typy}
\begin{itemize}
    \item \textbf{`xs:byte`, `xs:short`, `xs:long`, `xs:unsignedShort`, atd.} - Různé formáty číselných typů pro specifické potřeby, jako jsou velikosti a znaménka.
    \item \textbf{`xs:nonNegativeInteger`, `xs:positiveInteger`} - Pro nezáporná a kladná celá čísla.
\end{itemize}

\subsection{Speciální datové typy}
\begin{itemize}
    \item \textbf{`xs:ID`, `xs:IDREF`, `xs:ENTITY`, atd.} - Speciální účelové typy pro unikátní identifikátory, odkazy na identifikátory a jména entit.
\end{itemize}

\begin{figure}[H]
\centering
\begin{lstlisting}[
           language=xml,
           showspaces=false,
           basicstyle=\ttfamily,
           commentstyle=\color{gray},
           keywordstyle=\color{cyan}
]
<?xml version="1.0"?>
<xs:schema xmlns:xs="http://www.w3.org/2001/XMLSchema">
    <xs:element name="knihovna">
        <xs:complexType>
            <xs:sequence>
                <xs:element 
                    name="kniha" 
                    maxOccurs="unbounded">
                    <xs:complexType>
                        <xs:sequence>
                            <xs:element
                                name="nazev" 
                                type="xs:string"/>
                            <xs:element
                                name="autor" 
                                type="xs:string"/>
                            <xs:element 
                                name="vydano" 
                                type="xs:date"/>
                        </xs:sequence>
                        <xs:attribute 
                            name="id" 
                            type="xs:string" 
                            use="required"/>
                    </xs:complexType>
                </xs:element>
            </xs:sequence>
        </xs:complexType>
    </xs:element>
</xs:schema>

\end{lstlisting}
\caption{Ukázka XSD souboru}
\label{fig:xsd_file}
\end{figure}
