\chapter{Postata DOM pro XML a HTML}
\label{cha:PostataDom}

\begin{enumerate}
\item \textbf{Hierarchická struktura:} DOM transformuje celý HTML nebo XML dokument do stromové struktury, kde každý uzel stromu reprezentuje část dokumentu, jako jsou elementy, text, komentáře a atributy. Tento strom umožňuje snadný přístup a manipulaci s jednotlivými uzly.

\item \textbf{Programové rozhraní:} DOM poskytuje univerzální rozhraní, díky kterému mohou programy a skripty dynamicky interagovat s dokumentem. Například, JavaScript může přidávat, odstraňovat nebo měnit elementy, reagovat na události uživatelů, měnit styly a tak dále.

\item \textbf{Nezávislost na platformě a jazyku:} Ačkoli DOM je úzce spojen s webovými prohlížeči a JavaScriptem, je to standardizovaná technologie, kterou spravuje W3C (World Wide Web Consortium) a je nezávislá na platformě nebo programovacím jazyku. To znamená, že jakýkoliv jazyk, který podporuje DOM (např. Python s knihovnou pro práci s webovými dokumenty), může manipulovat s HTML nebo XML dokumenty.

\item \textbf{Dynamické změny dokumentu:} DOM umožňuje skriptům provádět změny v dokumentu "za běhu", což znamená, že změny se projeví ihned v prohlížeči bez potřeby znovu načítat stránku. Toto je zásadní pro dynamické webové aplikace, jako jsou jednostránkové aplikace (SPA), které se spoléhají na rychlé a plynulé uživatelské rozhraní.

\item \textbf{Podpora událostí:} DOM definuje, jak jsou události zpracovány v dokumentech. Skripty mohou reagovat na události uživatelů, jako jsou kliknutí myši, stisknutí kláves, pohyby myši atd., což umožňuje vytvářet interaktivní webové stránky.
\end{enumerate}