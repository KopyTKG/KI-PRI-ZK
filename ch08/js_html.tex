\chapter{Javascript a HTMLdom}
\label{cha:Javascript_a_HTMLdom}
Podstat JavaScriptu v HTMLdom je totožná s XMLdom viz.~\ref{cha:Javascript_a_XML_DOM}

\section{K čemu je to potřeba}
JavaScript společně s HTML DOM se využívá pro vytváření interaktivních a dynamických webových aplikací. Umožňuje reagovat na uživatelské akce, jako jsou kliknutí myši a stisky kláves, a dynamicky aktualizovat obsah bez nutnosti znovunačítat stránku. To zlepšuje uživatelskou zkušenost a efektivitu webových aplikací.

\section{Kdo to spravuje}
HTML DOM je standardizovaný a spravovaný World Wide Web Consortium (W3C), mezinárodní organizací, která udržuje a vyvíjí webové standardy, aby zajistila dlouhodobý růst a srozumitelnost internetu.


\section{Hierarchie HTML DOM}

HTML DOM reprezentuje strukturu HTML dokumentu jako strom uzlů, kde každý uzel může být:
\begin{itemize}
  \item \textbf{Document} - Reprezentuje celý dokument.
  \item \textbf{Element} - Reprezentuje HTML elementy jako jsou <div>, <p>, <a> atd.
  \item \textbf{Text} - Obsahuje text uvnitř elementů.
  \item \textbf{Attribute} - Reprezentuje atributy elementů, jako je 'class', 'id', atd.
\end{itemize}

\section{Manipulace s uzly}

JavaScript může manipulovat s HTML DOM pomocí různých metod a vlastností, umožňující:
\begin{itemize}
  \item Přidávat nové uzly.
  \item Odstraňovat stávající uzly.
  \item Měnit obsah a atributy uzlů.
\end{itemize}
